\documentclass[a4paper]{article}

\usepackage{amsmath}
\usepackage{amssymb}

\title{Theory for the \texttt{Structure} Class}
\author{A Hartloper}

\newcommand{\structureClass}{\texttt{Structure}}
\newcommand{\componentClass}{\texttt{Component}}

\begin{document}
\maketitle

\section{Introduction}

This document outlines the theory for the various parts of the \structureClass{} class.
The purpose of making this outline is to make a precise definition that will guide the implementation.
We start with the basic parts of the \structureClass{}, and build-up until we have enough to define the \structureClass{} itself.

\section{Elements of the \structureClass{}}

\subsection{\componentClass{}s}

\structureClass{}s are built up from \componentClass{}s that represent physical components in a structural system.
The definition of the n-tuple component, $\mathcal{C}$, is 
\begin{equation}
    \mathcal{C} = (O, S, B, C),
\end{equation}
where:
\begin{itemize}
    \item the local coordinate system is $O = (o, n_1, n_2, n_3)$, $o = (o_1, o_2, o_3) \in \mathbb{R}^3$ is the origin of the local coordinates, and $n_i \in \mathbb{R}^3$ ($i = 1, 2, 3$) are the vectors that define the orientation;
    % todo: needs a definition for the continuum domain
    \item the continuum nodes relative to the local coordinate system, $S = \{p \mid p \in \Omega^c\}$, where $\Omega^c$ is the continuum domain for the component;
    \item the beam nodes relative to the local coordinate system, $B = \{p \mid p \in \Omega^b\}$, where $\Omega^b$ is the beam domain for the component;
    \item the couplings, $C = \{c_j\}_{j=1}^{n_c}$, and $c_j = \{(p^b, P^s) \mid p^b \in B \cap \Gamma_j, P^s = \{ p^s \mid p^s \in S \cap \Gamma_j \}  \}$, where $\Gamma_j$ is an interface between beam and shell domains. Note that there is only one node in the set $B \cap \Gamma_j$, and there can be many nodes in the set $S \cap \Gamma_j$.
    \item Furthermore, the two domains are disjoint, i.e., $\Omega^c \cap \Omega^b = \emptyset$.
\end{itemize}


Essentially, a \componentClass{} consists of two domains: the continuum representing solid or shell elements, and the beam representing beam elements.
Each domain is likely disjoint.
These two domains are defined by nodes that have coordinates relative to a local coordinate system.
Couplings represent the link between these two domains.
The definition of this local coordinate system is also a part of the \componentClass{} itself.
This definition is useful because we can define functions that operate on the local coordinate system of the component, and then arrange the components back into the global system that defines the \structureClass{}.

\subsection{Imperfections}

Imperfections are defined as a set of functions that are applied to the \componentClass{}s.


\end{document}